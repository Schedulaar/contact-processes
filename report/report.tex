\documentclass[a4paper,11pt]{amsart} 
\usepackage[top = 2.5cm, bottom = 2.5cm, left = 2.5cm, right = 2.5cm]{geometry} 

\usepackage[T1]{fontenc}
\usepackage{hyperref}
\usepackage[utf8]{inputenc}
\usepackage{algorithm,algpseudocode}
\usepackage{float}
\usepackage{amsthm}
\usepackage{amsmath}
\usepackage{amsaddr}
\usepackage{xcolor}
\usepackage{mathtools}
\usepackage{tikz}
\usepackage{pgfplots}
\usepackage{multirow} 
\usepackage{booktabs}
\usepackage{graphicx}

\usepackage{setspace}
\usepackage{float}
\usepackage{fancyhdr}
\usepackage{mathtools}

\pagestyle{fancy}
\fancyhf{}

\setlength{\parindent}{0in}
\usepackage{etoolbox}

\makeatletter
\patchcmd{\@settitle}{\uppercasenonmath\@title}{}{}{}
\makeatother



\title{\href{https://schedulaar.github.io/contact-processes}{THE CONTACT PROCESS}\\
{\footnotesize INTERACTING PARTICLE SYSTEMS}}
\author{Michael Markl}
\email{michael.markl@uni-a.de}
\date{December 8, 2020}

\theoremstyle{theorem}
\newtheorem{theorem}{Theorem}
\newtheorem{proposition}[theorem]{Proposition}
\newtheorem{corollary}[theorem]{Corollary}

\theoremstyle{definition}
\newtheorem{definition}[theorem]{Definition}
\bibliographystyle{plain}

\newcommand{\todo}[1]{\textcolor{red}{#1}}
\newcommand{\Cont}{C}
\newcommand{\diff}{\,\mathrm{d}}
\newcommand{\Z}{\mathbb{Z}}
\renewcommand{\P}{\mathbb{P}}

\DeclareMathOperator{\LGen}{\mathcal{L}}

\DeclarePairedDelimiter\abs{\lvert}{\rvert}%
\DeclarePairedDelimiter\trin{\lvert\lvert\lvert}{\rvert\rvert\rvert}%
\DeclarePairedDelimiter\ceil{\lceil}{\rceil}
\DeclarePairedDelimiter\floor{\lfloor}{\rfloor}

\begin{document}

\thispagestyle{empty}
\maketitle

\section{Introduction and Preliminaries}

Contact processes are special spin-flip systems which can used to model, for example, the spread of an infection.
In the general case, we are given a countable, undirected and connected graph $G=(S,E)$ of bounded degree $\deg_{\max}\coloneqq \sup_{x\in S} \deg(x) < \infty$.
The nodes of the graph are usually called \emph{sites} and during the contact process sites are either \emph{infected} or \emph{healthy}.
Hence, the state space of the process we are about to define is $\Omega\coloneqq \{0, 1\}^S$ where $0$ should be interpreted as healthy and $1$ as infected.

We denote sites by letters $x,y \in S$, configurations by $\eta, \zeta, \xi\in\Omega$.
The resulting configuration for flipping site $x$ in configuration is denoted as $\eta^x$ and two sites $x$ and $y$ are called \emph{neighboring} ($x \sim y$) if $\{x,y\}$ is an edge in $E$.
We often use $\eta$ as the set consisting of all sites $x$ with $\eta(x)=1$.

In a contact process, an infected site becomes healthy after a unit exponential time.
On the contrary a healthy site becomes infected at a rate proportional to the number of infected neighbors.
This proportionality coefficient $\lambda > 0$ is independent of the site itself.
Now we can define the flip rates of a site $x$ in a configuration $\eta$ by $$c(x,\eta) \coloneqq \begin{cases}
	1 & \text{if $\eta(x) = 1$,} \\
	\lambda \cdot \abs{ \{ y\sim x \mid \eta(y) = 1 \} } & \text{if $\eta(x) = 0$.}
\end{cases}$$

As any other spin system, these spin rates can be translated into a continuous time Markov chain with $Q$-Matrix of the form $q(\eta, \eta^x) \coloneqq c(x,\eta)$ and generator
$$
\LGen f(\eta) \coloneqq \sum_{x\in S} c(x,\eta) \left( f(\eta^x) - f(\eta) \right),
$$
if $M\coloneqq \sup_{x\in S} \sum_{u:x\neq u} \gamma(x,u) < \infty$ holds where $\gamma(x,u)\coloneqq \sup_{\eta\in\Omega} \abs{ c(x,\eta^u) - c(x,\eta) }$.
Here, this is indeed the case:
For $u\sim x$ we have $\gamma(x,u)=\lambda$, otherwise $\gamma(x,u)$ vanishes implying $M = \lambda \cdot \deg_{\max}$.
By looking at the $(M<\epsilon)$-Theorem with $$\epsilon\coloneqq \inf_{x\in S, \eta\in \Omega} c(x,\eta)+ c(x,\eta^x) = 1,$$ we get the following result:
If  $\lambda <\deg_{\max}^{-1}$, then $\eta_t$ is \emph{ergodic}, i.e. there is a unique stationary distribution $\mu$ and for every $\eta\in\Omega$ and $f\in \Cont(\Omega)$ it fulfills $\lim_{t\to\infty} S_t f(\eta) = \int_\Omega f \diff \mu,$
where $S_t$ denotes the semigroup generated by $\LGen$.
As the pointmass $\delta_0$ on $0$ is always an invariant measure, we discuss in the next sections whether there are more invariant measures for $\lambda \geq \deg_{\max}^{-1}$.


Before that, we note, that any contact process is an \emph{attractive spin system}: If $\eta \leq \zeta$ holds component-wise, we have $c(x,\eta) = \lambda \cdot \abs{ \{ y\sim x\mid \eta(y) = 1 \}}\leq \lambda \cdot \abs{\{ y\sim x\mid \zeta(y) = 1 \}} = c(x,\zeta)$ for $\eta(x) = 1$ and $c(x, \eta) = 1 \leq 1 = c(x,\zeta)$ for $\zeta(x)=0$.
From the theory of attractive spin systems we know the existence of a lower invariant measure $\underline{\nu}\coloneqq \lim_{t\to\infty} \delta_0 S_t$
and an upper invariant measure $\overline{\nu} \coloneqq \lim_{t\to\infty} \delta_1 S_t$.
As $\delta_0$ is already invariant, $\underline{\nu} = \delta_0$ follows immediately.
The structure of $\overline{\nu}$ is less obvious for $\lambda \geq \deg_{\max}^{-1}$ as we will see in the next sections.


\section{To Be or Not to Be}
A central aspect we want to analyze is whether an infection goes extinct at some time in the future or survives ``until the end of time''.
In fact, there are two notions of survival discussed:

\begin{definition}[Survival]
	We say, a contact process \emph{survives (weakly)} if there is an $x\in S$ such that $\eta \coloneqq \{ x \}$ fulfills $$\P_{\{x\}}(\forall t \geq 0: \eta_t \neq \emptyset) > 0.$$
	We say it \emph{strongly survives} if there is an $x\in S$ such that $\eta\coloneqq \{ x \}$ fulfills $$\P_{\{ x \}}( x \in \eta_t \text{ for a sequence of $t$'s increasing to $\infty$}) > 0.$$
	Otherwise, the process \emph{dies out}.
\end{definition}

We note, that if the probability of weak survival above is positive for some $x\in S$, then it is positive for any $y\in S$ or even for any finite $\eta \in \Omega$ as the graph is assumed to be connected.

We are interested in the possible values $\lambda$ can attain, such that the process weakly or strongly survives.
Using a simple coupling argument we can show that increasing the infection rate of a surviving process results in a surviving process and decreasing the infection rate for a dying process results in a dying process.

\begin{proposition}[Critical Values]
	There are values $\lambda_c \leq \lambda_s$ in $[0,\infty]$ such that the process will survive strongly, if $\lambda > \lambda_s$, survive weakly, if $\lambda > \lambda_c$, or die out, if $\lambda < \lambda_c$.
	We call $\lambda_c$ and $\lambda_s$ the \emph{critical values}.
\end{proposition}


Immediately, multiple questions arise:
Are $\lambda_c$ and $\lambda_s$ positive? Is it possible that a process will never survive independent of the infection rate?
Is there a graph with $\lambda_c < \lambda_s$, such a process might survive only weakly?
The first of these question will be addressed using the graphical representation and the self-duality of contact processes:

\begin{figure}[h]
\begin{tikzpicture}
	\draw[line width=2pt] (-2.7,0) -- (2.7,0);
	
	\foreach \x in {-2, -1, ..., 2} {
		\node at (\x,0) [below] {\x};
		\draw (\x,0) -- (\x, 5);
	}
	
	\fill (-2,1.2) circle (1pt) node [right] {$\delta$};
	\fill (-2,4.2) circle (1pt) node [right] {$\delta$};
	
	\fill (-1,0.8) circle (1pt) node [right] {$\delta$};
	\fill (-1,3.2) circle (1pt) node [right] {$\delta$};
	
	\fill (1,1.1) circle (1pt) node [right] {$\delta$};
	\fill (1,3.3) circle (1pt) node [right] {$\delta$};
	
	\fill (2,2.9) circle (1pt) node [right] {$\delta$};
	
	\draw[->] (-2, 3.9) -> (-1,3.9);
	\draw[->] (-1, 2.7) -> (-2,2.7);
	\draw[->] (-1, 1.0) -> (-2,1.0);
	\draw[->] (-1, 4.4) -> (0,4.4);
	\draw[->] (-1, 1.4) -> (0,1.4);
	\draw[->] (0, 2.2) -> (-1,2.2);
	\draw[->] (1, 3.6) -> (0,3.6);
	\draw[->] (1, 2.5) -> (2,2.5);
	
	\node at (-2.7, 5) {$t$};
\end{tikzpicture}
\caption{Graphical Representation of the contact process}
\label{fig:graphical-rep}
\end{figure}

The contact process can be displayed very intuitively in a \emph{graphical representation}:
An example can be seen in~\autoref{fig:graphical-rep} for the integer lattice $\Z$:
We use the \emph{space-time} picture on $S\times [0,\infty)$. For each site, we generate \emph{heal events} at a unit exponential rate which are denoted as $\delta$ in the graph.
Additionally we create \emph{infection events} in the form of arrows for each directed edge in the graph at rate $\lambda$.
Then we call a path from $(x,s)$ to $(y,t)$ in this picture \emph{active}, if it only walks upwards in time at sites or along an infection arrow and never passes a heal event.
Given an initial infected set $\eta$ at time $0$, we can deduce the set of infected sites at time $t$ as $$\eta_t = \{y\in S\mid \exists\ \text{active path from $(x,0)$ to $(y,t)$ for some $x\in\eta$}\}.$$


\begin{proposition}[Self-Duality]
	For any $\eta, \zeta\in \Omega$, we have $\P_\eta(\eta_t \cap \zeta \neq \emptyset) = \P_\zeta(\eta \cap \zeta_t \neq \emptyset)$.
\end{proposition}
\begin{proof}
	We fix some graphical representation of the process and assume, that if we start the process with infected set $\eta$, there is a site in $\zeta$ that is infected at time $t$.
	The dual process is constructed as follows:
	We look at all possible sites that could have led to an infection state at any site of $\zeta$ at time $t$.
	This can be done by traversing the graphical representation backwards:
	We collect in $\hat{\zeta}_{t-s}$ all sites $x\in S$ for which there exists an active path from $(x,s)$ to $(y,t)$ for some $y\in \zeta$.
	As the rate of infection events from a site $x$ to a site $y$ is the same as for the infection rate from $y$ to $x$, we observe, that $\hat{\zeta}_{t-s}$ is by distribution equal to $\zeta_s$.
	
	Finally, we observe, that if the infection starting from $\eta$ leads to an infected site of $\zeta$ at time $t$, then $\hat{\zeta}_0$ must contain an element of $\eta$ implying $\P_\eta(\eta_t \cap \zeta \neq \emptyset) \leq \P_\zeta(\eta \cap \zeta_t \neq \emptyset)$.
\end{proof}
Applying the duality relation above to finite $\eta$ and all nodes $S$, we get for $t$ approaching $\infty$:
\begin{corollary}
For finite $\eta\in\Omega$ we have $\P_\eta(\forall t\geq0: \eta_t\neq\emptyset) = \overline{\nu}(\{ \zeta\in\Omega \mid \eta\cap\zeta\neq\emptyset \}).$

Moreover, the process dies out iff $\delta_0$ is the only invariant measure.
Hence $\lambda_c \geq \deg_{\max}^{-1} > 0$.
\end{corollary}

\section{The Contact Process on Homogeneous Trees}
Here, $G$ is a homogeneous tree, i.e. a graph in which each node has $d+1$ neighboring nodes.
We discuss two main theorems.
The first gives bounds for $\lambda_c$, in particular implying the possibility of weak survival:
\begin{theorem}
	The contact process on a homogeneous tree satisfies
	$$\frac{1}{d+1} \leq \lambda_c \leq \frac{1}{d-1}$$
\end{theorem}

The second result shows that for large enough $d$, there might be weak without strong survival:
\begin{theorem}
	The critical value for strong survival on the homogeneous tree satisfies $$\lambda_s \geq \frac{1}{2\sqrt{d}}.$$
	Therefore $\lambda_s > \lambda_c$ holds for $d\geq 6$.
\end{theorem}

\section{Results for Integer Lattices}

The same technique for showing weak survival works for the integer lattice $\Z$, but finding a superharmonic function $f$ and value $\rho$ that satisfy the necessary properties is more involved. Nevertheless, this yields the result, that $\Z$ (and hence any graph containing $\Z$) has critical value $\lambda_c\leq 2$.
More generally, for a multidimensional integer lattice $\Z^d$ one can bound $\lambda_c$ by $2/d$.


\end{document}