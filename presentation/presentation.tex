
\documentclass[envcountsect, aspectratio=149]{beamer}
\usetheme[headheight=0em, logo=none, themepath=beamercolorthemefibeamer, nofonts]{fibeamer}
\usepackage{beamercolorthemefibeamer}
\usepackage[utf8]{inputenc} % for input encoding
\usepackage[german]{babel} % for german localization
\usepackage{colonequals} % for :=
\usepackage{graphicx} % for \includegraphics
\usepackage{wrapfig} % for floating images to the right
\usepackage{tikz}
\usepackage{esvect}
\usepackage{tabto}
\usepackage{enumitem}
\usepackage{mathabx}
\usepackage{mathtools}
\usepackage{framed}
\usepackage{tikz}

\usetikzlibrary{shapes,snakes}
\usetikzlibrary{arrows.meta}
\usetikzlibrary{calc}

%-------------------------------------------------------------------------------
% Hilfreiche Befehle
%-------------------------------------------------------------------------------
\newcommand{\betrag}[1]{\lvert #1 \rvert}	        % Betrag
\providecommand*{\Lfloor}{\left\lfloor}                 % gro\ss{}es Abrunden
\providecommand*{\Rfloor}{\right\rfloor}                % gro\ss{}es Abrunden
\providecommand*{\Floor}[1]{\Lfloor #1 \Rfloor}         % gro\ss{}es ganzes Abrunden
\providecommand*{\Ceil}[1]{\left\lceil #1 \right\rceil} % gro\ss{}es ganzes Aufrunden

\newcommand{\Z}{\mathbb{Z}}
\newcommand{\N}{\mathbb{N}}
\newcommand{\R}{\mathbb{R}}
\newcommand{\Q}{\mathbb{Q}}
\newcommand{\firstNumbers}[1]{[#1]}
\newcommand{\transpose}{^\intercal}
\newcommand{\subjectTo}{\textbf{s.t.}}
\newcommand{\MIPR}{MIP\textsuperscript{*}}
\newcommand{\MIPI}{MIP}
\newcommand{\oBdA}{oBdA.}
\newcommand{\rang}{\operatorname{rang}}
\newcommand{\norm}[1]{\left\lVert#1\right\rVert_\infty}
\newcommand{\zero}{0}
\newcommand{\todo}[1]{{\color{red}{\emph{TODO: }}#1}}
\newcommand{\one}{\mathbbm{1}}
\newcommand{\eq}[1]{{\operatorname{eq}(#1)}}
\newcommand{\co}[1]{\operatorname{co}(#1)}


\makeatletter
\newcommand*{\da@rightarrow}{\mathchar"0\hexnumber@\symAMSa 4B }
\newcommand*{\da@leftarrow}{\mathchar"0\hexnumber@\symAMSa 4C }
\newcommand*{\xdashrightarrow}[2][]{%
  \mathrel{%
    \mathpalette{\da@xarrow{#1}{#2}{}\da@rightarrow{\,}{}}{}%
  }%
}
\newcommand{\xdashleftarrow}[2][]{%
  \mathrel{%
    \mathpalette{\da@xarrow{#1}{#2}\da@leftarrow{}{}{\,}}{}%
  }%
}
\newcommand*{\da@xarrow}[7]{%
  % #1: below
  % #2: above
  % #3: arrow left
  % #4: arrow right
  % #5: space left 
  % #6: space right
  % #7: math style 
  \sbox0{$\ifx#7\scriptstyle\scriptscriptstyle\else\scriptstyle\fi#5#1#6\m@th$}%
  \sbox2{$\ifx#7\scriptstyle\scriptscriptstyle\else\scriptstyle\fi#5#2#6\m@th$}%
  \sbox4{$#7\dabar@\m@th$}%
  \dimen@=\wd0 %
  \ifdim\wd2 >\dimen@
    \dimen@=\wd2 %   
  \fi
  \count@=2 %
  \def\da@bars{\dabar@\dabar@}%
  \@whiledim\count@\wd4<\dimen@\do{%
    \advance\count@\@ne
    \expandafter\def\expandafter\da@bars\expandafter{%
      \da@bars
      \dabar@ 
    }%
  }%  
  \mathrel{#3}%
  \mathrel{%   
    \mathop{\da@bars}\limits
    \ifx\\#1\\%
    \else
      _{\copy0}%
    \fi
    \ifx\\#2\\%
    \else
      ^{\copy2}%
    \fi
  }%   
  \mathrel{#4}%
}
\makeatother


\setbeamertemplate{theorems}[numbered]
\newtheorem{conjecture}[theorem]{Vermutung}
\newtheorem{korollar}[theorem]{Korollar}
\newtheorem{beispiel}[theorem]{Beispiel}
\newtheorem{proposition}[theorem]{Proposition}

\definecolor{darkblue}{HTML}{00446B}
\definecolor{darkgreen}{HTML}{006B44}

\newcommand{\bigO}{\mathcal{O}}
\DeclarePairedDelimiter\abs{\lvert}{\rvert}
\newcommand{\size}[1]{\langle#1\rangle}
\newcommand{\coloniff}{\vcentcolon\Longle ftrightarrow}



\newcommand{\Cont}{C}
\newcommand{\diff}{\,\mathrm{d}}
\newcommand{\E}{\mathbb{E}}
\renewcommand{\P}{\mathbb{P}}

\DeclareMathOperator{\LGen}{\mathcal{L}}


\DeclarePairedDelimiter\trin{\lvert\lvert\lvert}{\rvert\rvert\rvert}%
\DeclarePairedDelimiter\ceil{\lceil}{\rceil}
\DeclarePairedDelimiter\floor{\lfloor}{\rfloor}


\makeatletter
\makeatother
\useoutertheme{infolines}
\newenvironment{noheadline}{
	\setbeamertemplate{headline}{}
}{}
\newenvironment{nofootline}{
\setbeamertemplate{footline}{}
}{}

\setbeamersize{text margin left=2.7em, text margin right=2.7em}
\setbeamertemplate{frametitle}{\insertframetitle}
\setbeamercolor{block body}{fg=black!90}

\setbeamerfont{title}{size=\huge}
\setbeamertemplate{institute}{\insertinstitute}

\AtBeginSection{
	\begin{noheadline}
		\begin{frame}
		\vfill
		\centering
		\begin{beamercolorbox}[sep=8pt,center]{title}
			\usebeamerfont{title}\insertsectionhead\par%
		\end{beamercolorbox}
		\vfill
	\end{frame}
	\end{noheadline}

}

\makeatletter
\setbeamertemplate{footline}{%
	\color{darkblue}% to color the progressbar
	\small
	\hfill{\insertframenumber\hspace{1.5em}}
	\vspace{1em}


	\hspace*{-\beamer@leftmargin}%
	\rule{\beamer@leftmargin}{1pt}%
	\rlap{\rule{\dimexpr\numexpr0\insertframenumber\dimexpr
			\textwidth\relax/\numexpr0\inserttotalframenumber}{1pt}}
	% next 'empty' line is mandatory!

	\vspace{0\baselineskip}
	{}
}


\setbeamertemplate{bibliography item}{\insertbiblabel}

\title{\LARGE The Contact Process}
\subtitle{Interacting Particle Systems}
\author{~\\Michael Markl \\ December 8, 2020}
\date{23.01.2020}
\renewcommand{\[}{
	\setlength\abovedisplayskip{0.5ex}
	\setlength{\belowdisplayskip}{0.5ex}
	\setlength{\abovedisplayshortskip}{0.5ex}
	\setlength{\belowdisplayshortskip}{0.5ex}\begin{equation*}}

\institute{Insitut für Mathematik der Universität Augsburg\\Diskrete Mathematik, Optimierung und Operations Research}

\usefonttheme{professionalfonts}

\DeclareFontFamily{OML}{zavm}{\skewchar\font=127 }
\DeclareFontShape{OML}{zavm}{m}{it}{<-> s*[.85] zavmri7m}{}
\DeclareFontShape{OML}{zavm}{b}{it}{<-> s*[.85] zavmbi7m}{}
\DeclareFontShape{OML}{zavm}{m}{sl}{<->ssub * zavm/m/it}{}
\DeclareFontShape{OML}{zavm}{bx}{it}{<->ssub * zavm/b/it}{}
\DeclareFontShape{OML}{zavm}{b}{sl}{<->ssub * zavm/b/it}{}
\DeclareFontShape{OML}{zavm}{bx}{sl}{<->ssub * zavm/b/sl}{}
\DeclareMathAlphabet{\mathsf}{OML}{zavm}{m}{it} % not `n'

\DeclareFontFamily{OT1}{zavm}{\skewchar \font =127}
\DeclareFontShape{OT1}{zavm}{m}{n}{<-> s*[.85] zavmr7t}{}
\DeclareFontShape{OT1}{zavm}{b}{n}{<-> s*[.85] zavmb7t}{}
\DeclareFontShape{OT1}{zavm}{bx}{n}{<->ssub * zavm/b/n}{}

\DeclareFontFamily{OMS}{zavm}{}
\DeclareFontShape{OMS}{zavm}{m}{n}{<-> s*[.85] zavmr7y}{}

\SetSymbolFont{operators}   {normal}{OT1}{zavm}{m}{n}
\SetSymbolFont{letters}     {normal}{OML}{zavm}{m}{it}
\SetSymbolFont{symbols}     {normal}{OMS}{zavm}{m}{n}
\SetSymbolFont{largesymbols}{normal}{OMX}{iwona}{m}{n}

\DeclareMathAlphabet{\mathcal}{OMS}{cmsy}{m}{n}
\SetMathAlphabet{\mathcal}{bold}{OMS}{cmsy}{b}{n}


%\setlist[enumerate]{topsep=0.5ex,itemsep=0ex,partopsep=0ex,parsep=0.8ex}


\beamertemplatenavigationsymbolsempty

\begin{document}

	\setcounter{framenumber}{-1}

	\begin{nofootline}
		\frame{\titlepage}
	\end{nofootline}

	%\begin{frame}{Outline}\tableofcontents\end{frame}

	\section{What is a Contact Process?}
	\subsection{Introduction and Preliminaries}
	
	\begin{frame}
		Let $G=(S,E)$ be a connected graph with bounded degree $\deg_{\max}$.
		
		\pause\bigskip
				
		Flip rate for an infected set $\eta\in \mathcal{P}(S) \hat{=}\{0,1\}^S $ at site $x\in S$:
		$$c(x,\eta) \coloneqq \begin{cases}
			1, & \text{if $x\in\eta$},\\
			\lambda \cdot \abs{\{ y\in \eta \mid x\sim y \}}, & \text{if $x \notin \eta$.}
		\end{cases}
		$$
		
		\pause\bigskip
		
		Generator: $\displaystyle \LGen f(\eta) = \sum_{x\in S} c(x,\eta) \left( f(\eta^x) - f(\eta) \right)$
		
		\pause\bigskip
		
		\begin{align*}
		M &= \sup_{x\in S} \sum_{u:x\neq u} \gamma(x,u)\onslide<6->{ = \lambda \cdot \deg_{\max} } \\
		\gamma(x,u) &= \sup_{\eta\in\Omega}\, \abs{ c(x,\eta^u) - c(x,\eta) }\onslide<5->{ = \begin{cases}
			\lambda & \text{if $u\sim x$},\\
			0 & \text{if $u\nsim x$}.
		\end{cases}} \\[0.5em]
		\onslide<7-> {
			\epsilon &= \inf_{x\in S, \eta \in \Omega} c(x,\eta) + c(x,\eta^x)}
		\onslide<8->{ = 1}
		\end{align*}
	\end{frame}
	
	\begin{frame}{Invariant Measures}
		\pause
		A contact process is an \emph{attractive} spin system:
		If $\eta\subseteq \zeta$, then
			$$x \in \eta\, \Rightarrow\, c(x,\eta) \leq c(x,\zeta),\hspace{2em} x\notin\zeta \,\Rightarrow\, c(x,\eta) \geq c(x,\zeta) $$
			
		\bigskip\pause
		Hence, we have upper and lower invariant measures:
		$$\overline{\nu} \coloneqq \lim_{t\to\infty} \delta_1 S_t, \hspace{2em} \underline{\nu} \coloneqq \lim_{t\to\infty} \delta_0 S_t\pause = \delta_0 $$ 
		
		\bigskip\pause
		The $M<\epsilon$ criterion implies: If $\lambda < \deg_{\max}^{-1}$, then $\mathcal{I} = \{ \delta_0 \}$.
	\end{frame}
	
	\section{To Be or Not to Be}
	\subsection{Survival or Extinction}
	
	\begin{frame}{Survival or Extinction}
		\begin{definition}[Survival]
			A contact process $\eta(t)$ \emph{survives (weakly)} if there is an $x\in S$ such that $$\P_{\{x\}} (\forall t \geq0 : \eta_t \neq \emptyset) > 0.$$
			\pause
			It \emph{survives strongly} if there is an $x\in S$ such that
			$$\P_{\{x\}}(x\in\eta_t \text{ for a sequence $t$'s increasing to $\infty$})> 0.$$\pause
			Otherwise the process \emph{dies out}.
		\end{definition}
		\pause
		\begin{proposition}[Critical Values]
			There exist $\lambda_c \leq \lambda_s$ in $[0,\infty]$ such that the contact process
			\begin{itemize}[label=$\bullet$]
				\item survives weakly iff $\lambda > \lambda_c$,
				\item survives strongly iff $\lambda > \lambda_s$.
			\end{itemize}
		\end{proposition}
	\end{frame}
	
	\begin{frame}{Infection Process in Graphical Representation}
		\begin{columns}
		\begin{column}{0.48\textwidth}
			A path in $S\times [0,\infty)$ is \emph{active}, if
			\begin{itemize}[label=$\bullet$]
				\item it only walks upwards in time,
				\item it switches site only according to infection arrows,
				\item it does not pass any heal events.
			\end{itemize}
			\bigskip
			\onslide<4->{
			If $\eta$ is initially infected, then
			\vspace{-1em}\begin{align*}
				\eta_t = \{ x\in S \mid\, &\exists \text{ active path from $(x,0)$}\\
				&\text{to (y,t) for some $x\in\eta$} \}
			\end{align*}}
		\end{column}
		\begin{column}{0.48\textwidth}
			\begin{tikzpicture}
				
				\foreach \x in {-2, -1, ..., 2} {
					\node at (\x,0) [below] {\x};
					\draw (\x,0) -- (\x, 5);
				}
				
				\draw[->] (-2, 3.9) -> (-1,3.9);
				\draw[->] (-1, 2.7) -> (-2,2.7);
				\draw[->] (-1, 1.0) -> (-2,1.0);
				\draw[->] (-1, 4.4) -> (0,4.4);
				\draw[->] (-1, 1.4) -> (0,1.4);
				\draw[->] (0, 2.2) -> (-1,2.2);
				\draw[->] (0, 2.7) -> (1,2.7);
				\draw[->] (1, 3.6) -> (0,3.6);
				\draw[->] (1, 2.5) -> (2,2.5);
				
				\node at (-2.7, 5) {$t$};
				\onslide<2->{
				\draw[red, line width=1pt] (0,0) -- (0,2.2) -- (-1,2.2) -- (-1,2.7) -- (-2,2.7) -- (-2,3.9) -- (-1,3.9) -- (-1,4.4) -- (0,4.4) --(0,5.0);}
				\onslide<3->{
				 \draw[red, line width=1pt] (0,0) -- (0,2.2) -- (-1,2.2) -- (-1,2.7) -- (-2,2.7) -- (-2,3.9) -- (-1,3.9) -- (-1,4.4) -- (0,4.4) --(0,5.0);
				 \draw[red, line width=1pt] (0,0) -- (0,5);
				 \draw[red, line width=1pt] (-1, 2.2) -- (-1,3.2);
				 \draw[red, line width=1pt] (-1,3.9)--(-1,5);
				 \draw[red, line width=1pt] (0, 2.7)--(1,2.7)--(1,3.8);
 				 \draw[red, line width=1pt] (1, 3.6)--(0,3.6);
				}
				
								
				\fill (-2,1.2) circle (1pt) node [right] {$\delta$};
				\fill (-2,4.2) circle (1pt) node [right] {$\delta$};
				
				\fill (-1,0.8) circle (1pt) node [right] {$\delta$};
				\fill (-1,3.2) circle (1pt) node [right] {$\delta$};
				
				\fill (1,1.1) circle (1pt) node [right] {$\delta$};
				\fill (1,3.8) circle (1pt) node [right] {$\delta$};
				
				\fill (2,2.9) circle (1pt) node [right] {$\delta$};
				
				\draw[line width=2pt] (-2.7,0) -- (2.7,0);
			\end{tikzpicture}
		\end{column}
		\end{columns}
	\end{frame}
	
	\begin{frame}
			\addtocounter{theorem}{-1}
			\begin{proposition}[Critical Values]
				There exist $\lambda_c \leq \lambda_s$ in $[0,\infty]$ such that the contact process
				\begin{itemize}[label=$\bullet$]
					\item survives weakly iff $\lambda > \lambda_c$,
					\item survives strongly iff $\lambda > \lambda_s$.
				\end{itemize}
			\end{proposition}
			\emph{Proof.}
			\pause
			We show $\P_{\{x\}}(y\in \eta_t)$ is non-decreasing in $\lambda$ by a coupling argument.
			
			\pause Let $\eta(t)$, $\zeta(t)$ be contact processes with $\lambda_{\eta} \leq \lambda_{\zeta}$.
			
			\begin{columns}
			\begin{column}{0.48\textwidth}
				\begin{enumerate}[label=\arabic*)]
					\onslide<4->{\item Put $\delta$'s at rate $1$.}
					\onslide<5->{\item Put $\xrightarrow{\ \ \ \,}$~\,at rate $\lambda_\eta$.}
					\onslide<6->{\item Put $\xdashrightarrow{}$ at rate $(\lambda_\zeta - \lambda_\eta)$.}
				\end{enumerate}
			\end{column}
			\begin{column}{0.48\textwidth}
			\begin{tikzpicture}
				\draw[line width=2pt] (-2.7,0) -- (2.7,0);
				
				\foreach \x in {-2, -1, ..., 2} {
					\draw (\x,0) -- (\x, 3.8);
				}
				
				\onslide<4->{
				\fill (-2,1.2) circle (1pt) node [right] {$\delta$};
				
				\fill (-1,0.8) circle (1pt) node [right] {$\delta$};
				\fill (-1,3.2) circle (1pt) node [right] {$\delta$};
				
				\fill (1,1.1) circle (1pt) node [right] {$\delta$};
				\fill (1,3.3) circle (1pt) node [right] {$\delta$};
				
				\fill (2,2.9) circle (1pt) node [right] {$\delta$};
				}
				\onslide<5->{
				\draw[->] (-1, 2.7) -> (-2,2.7);
				\draw[->] (-1, 1.0) -> (-2,1.0);

				\draw[->] (0, 2.2) -> (-1,2.2);
				\draw[->] (1, 2.5) -> (2,2.5);
				}
				\onslide<6->{
				\draw[->, dashed] (-2, 3.6) -> (-1,3.6);
				\draw[->, dashed] (-1, 1.4) -> (0,1.4);
				\draw[->, dashed] (1, 3.6) -> (0,3.6);
				}
			\end{tikzpicture}
			
			\end{column}
			\end{columns}
			\vspace{0.5em}
			\qed
	\end{frame}
	
	\begin{frame}{Self Duality}
	\begin{proposition}[Self Duality]
		$\P_\eta(\eta_t \cap \zeta \neq \emptyset) = \P_\zeta(\eta\cap\zeta_t \neq \emptyset)$ holds for any $\eta,\zeta\in\{0,1\}^S$.
	\end{proposition}
	\pause 
	\begin{columns}
	\begin{column}{0.54\textwidth}
	\emph{Proof.}
	Construct dual process $\hat{\zeta}$ by
	\vspace{-0.5em}
	\begin{align*}
		\hat{\zeta}_{t-s} = \{ x\in S \mid\,& \exists\text{ active path from $(x,s)$} \\[-0.3em]
		& \text{to $(y,t)$ for some $y\in \zeta$} \}
	\end{align*}
	
	\onslide<4->{
	Now $\eta_t \cap \zeta \neq \emptyset \iff \hat{\zeta}_0 \cap \eta \neq \emptyset$.
	}
	
	\onslide<5->{
	\bigskip
	$\hat{\zeta}_{t-s}$ is by distribution equal to $\zeta_s$.
	
	
		\qed
	}
	
	\vspace{5em}
	
	\end{column}
	\begin{column}{0.42\textwidth}
		\begin{tikzpicture}
			
			\foreach \x in {-2, -1, ..., 2} {
				\node at (\x,0) [below] {\x};
				\draw (\x,0) -- (\x, 5);
			}
			
			\draw[->] (-2, 3.9) -> (-1,3.9);
			\draw[->] (-1, 2.7) -> (-2,2.7);
			\draw[->] (-2, 1.0) -> (-1,1.0);
			\draw[->] (-1, 4.4) -> (0,4.4);
			\draw[->] (-1, 1.4) -> (0,1.4);
			\draw[->] (0, 2.2) -> (-1,2.2);
			\draw[->] (0, 2.7) -> (1,2.7);
			\draw[->] (1, 3.6) -> (0,3.6);
			\draw[->] (2, 2.5) -> (1,2.5);
			
			\node at (-2.3, 5) {$t$};

			\onslide<3->{
		 	\draw[red, line width=1pt] (-1, 5) -- (-1,3.2);
		 	\draw[red, line width=1pt] (-1, 3.9) -- (-2,3.9) -- (-2,1.5);
		 	\draw[red, line width=1pt] (-2,2.7)--(-1,2.7)--(-1,0.8);
		 	\draw[red, line width=1pt] (-1,1.0)--(-2,1.0)--(-2,0);
			\draw[red, line width=1pt] (-1,2.2)--(0,2.2)--(0,0);
			\draw[red, line width=1pt] (-1,1.4)--(0,1.4);
			}
			
							
			\fill (-2,1.5) circle (1pt) node [right] {$\delta$};
			\fill (-2,4.2) circle (1pt) node [right] {$\delta$};
			
			\fill (-1,0.8) circle (1pt) node [right] {$\delta$};
			\fill (-1,3.2) circle (1pt) node [right] {$\delta$};
			
			\fill (1,1.1) circle (1pt) node [right] {$\delta$};
			\fill (1,3.8) circle (1pt) node [right] {$\delta$};
			
			\fill (2,2.9) circle (1pt) node [right] {$\delta$};
			
			\draw[line width=2pt] (-2.2,0) -- (2.2,0);
		\end{tikzpicture}
	\end{column}
	\end{columns}
	\end{frame}
	
	\begin{frame}
		\begin{corollary}
			For finite $\eta\in\{0,1\}^S$: $\P_\eta(\forall t\geq0: \eta_t \neq \emptyset) = \overline{\nu}(\{ \zeta\in \Omega \mid \eta \cap \zeta \neq \emptyset \})$
			\pause\medskip
			
			Hence, the process dies iff $\mathcal{I} = \{ \delta_0 \}$. Therefore, $\lambda_c\geq \deg_{\max}^{-1} > 0$.
		\end{corollary}
	\end{frame}
	
	
	\section{The Contact Process on Homogeneous Trees}
	\subsection{Weak $\neq$ Strong}
	
	\begin{frame}{Homogeneous Tree}
	\centering
	\begin{tikzpicture}[every node/.style={scale=0.6}]
	
	\node[circle, draw=black, inner sep=2pt] (r) at (0,0) {-1};
	
	\node[circle, draw=black, inner sep=2pt] (0) at (1,-1) {0};
	\node[circle, draw=black, inner sep=2pt] (1) at (1,1) {0};
	
	\node[circle, draw=black, inner sep=2pt] (00) at (2,-1.5) {1};
	\node[circle, draw=black, inner sep=2pt] (01) at (2,-0.5) {1};
	\node[circle, draw=black, inner sep=2pt] (10) at (2,0.5) {1};
	\node[circle, draw=black, inner sep=2pt] (11) at (2,1.5) {1};
	
	\node[circle, draw=black, inner sep=2pt] (000) at (3,-1.75) {2};
	\node[circle, draw=black, inner sep=2pt] (001) at (3,-1.25) {2};
	\node[circle, draw=black, inner sep=2pt] (010) at (3,-0.75) {2};
	\node[circle, draw=black, inner sep=2pt] (011) at (3,-0.25) {2};
	\node[circle, draw=black, inner sep=2pt] (100) at (3,0.25) {2};
	\node[circle, draw=black, inner sep=2pt] (101) at (3,0.75) {2};
	\node[circle, draw=black, inner sep=2pt] (110) at (3,1.25) {2};
	\node[circle, draw=black, inner sep=2pt] (111) at (3,1.75) {2};
	
	\draw (r) -- (0);
	\draw (r) -- (1);
	
	\draw (0) -- (00);
	\draw (0) -- (01);
		
	\draw (1) -- (10);
	\draw (1) -- (11);
	
		
	\draw (00) -- (000);
	\draw (00) -- (001);
			
	\draw (01) -- (010);
	\draw (01) -- (011);
				
	\draw (10) -- (100);
	\draw (10) -- (101);
					
	\draw (11) -- (110);
	\draw (11) -- (111);
	
	\end{tikzpicture}
	
		
				\bigskip
				
		Every node has degree $d+1$.
		
	\end{frame}
	
	
	\begin{frame}{Results for the Homogeneous Tree}
		\begin{theorem}[Weak Survival]
			On a homogeneous tree, we have
			$$\frac{1}{d+1} \leq \lambda_c \leq \frac{1}{d-1}$$
		\end{theorem}
		\pause
		\begin{theorem}[Bound for strong survival]
			On a homogeneous tree, we have
			$$\frac{1}{d+1} \leq \lambda_c \leq \frac{1}{d-1}$$
			In particular, $\lambda_c < \lambda_s$ for $d\geq 6$.
		\end{theorem}
	\end{frame}
	
	
	\begin{frame}{Important Tools}
		\begin{definition}[Superharmonicity]
		A function $f:\{0,1\}^S \to \R$ is \emph{superharmonic}, if $\E_\eta \abs{f(\eta_t)} < \infty$ and $\E_\eta f(\eta_t) \leq f(\eta)$ hold for all $t\geq0$ and $\eta\in\{0,1\}^S$.
		\end{definition}
		\pause
		\begin{proposition}
			If a nonconstant bounded superharmonic function $f$ satisfies $f(\emptyset) \geq f(\eta)$ for all $\eta\in\{0,1\}^S$, the process survives weakly.
		\end{proposition}
		\pause
		\begin{proposition}
			If $f:\{0,1\}^S \to \R$ is bounded and
			$\left. \frac{\diff}{\diff t} \E_\eta f(\eta_t) \right|_{t=0} \leq 0,$
			then $\E_\eta f(\eta_t)$ is decreasing in $t$.
		\end{proposition}
	\end{frame}
	
	\begin{frame}{Sketch of proof for $\lambda_c \leq (d-1)^{-1}$}
		Define $f(\eta)\coloneqq \rho^{\abs{\eta}}$ for some $\rho\in (0,1)$.
		
		\pause\medskip
		Calculate using $E_\eta\coloneqq \left\{ \{x,y\}\in E\mid x\in\eta, y\notin \eta \right\}$:
		\begin{align*}
		\left. \frac{\diff}{\diff t} \E_\eta f(\eta_t) \right|_{t=0} &= (\rho^{-1} \abs{\eta} - \lambda \cdot \abs{E_\eta}) (1-\rho) f(\eta) \\ \onslide<4->{&\leq \abs{\eta}(\rho^{-1} - \lambda(1-d))(1-\rho)f(\eta)}
		\end{align*}
		
		\pause
		
		Bound $\abs{\E_\eta} \geq \abs{\eta} \cdot(d+1) - 2\cdot(\abs{\eta} - 1) = \eta(d-1) + 2 \geq \abs{\eta} (d-1)$
		\pause\pause\medskip
		
		For $\displaystyle \rho=\frac{1}{\lambda (1-d)}$, $f$ is superharmonic due to Proposition 3.4.
		\pause\medskip
		
		Proposition 3.5 yields weak survival.
		
		\qed
\end{frame}
	
	\begin{noheadline}
		\begin{frame}<presentation:0>[noframenumbering]
			\cite{Liggett2010}
			\cite{Liggett1999}
		\end{frame}
	
		\begin{frame}{Literature}
			\scriptsize
			\bibliographystyle{acm}
			\bibliography{literature}
		\end{frame}
	\end{noheadline}

\end{document}
